\documentclass{article}
\usepackage[a4paper, margin=3cm, top=2cm]{geometry}
\usepackage{graphicx}
\usepackage{hyperref}
\usepackage{comment}
\usepackage{amsmath}
\usepackage{mathtools}
\usepackage{amssymb}
\usepackage{mathpartir}
\usepackage{enumitem}
\usepackage{titling}
\usepackage{amsthm}

% Title formatting
\pretitle{\centering\LARGE\bfseries\vspace{0em}}
\posttitle{\par\vspace{0em}}

\preauthor{\centering\large}
\postauthor{\par\vspace{0em}}

\predate{\centering\small}
\postdate{\par}


\setlength{\parindent}{0pt} % no indend automatically everywhere

\title{Small step semantics for arithmetic expressions in Lean}
\author{}
\date{} % empty date



\begin{document}
% theorem environments
\newtheorem{theorem}{Theorem}
\newtheorem{lemma}{Lemma}

\maketitle
In this document we will define small step rules for the datatype defined in \texttt{math-expr.lean}. After that, we will show that the \texttt{eval} function defined in Lean corresponds to the defined small step rules. %The proof are paper-proofs, but also implemented in Lean.

\section{Rules}
The datatype is defined as 
\[
e ::= \operatorname{const}(q) \mid \operatorname{var}(x) \mid \operatorname{add}(e_1,e_2) \mid \operatorname{sub}(e_1,e_2) \mid \operatorname{mul}(e_1,e_2) \mid \operatorname{div}(e_1,e_2)\qquad q\in\mathbb{Q},~x\in\text{String}.
\]
An \emph{environment} $\Gamma : \text{Var} \rightharpoonup \mathbb{Q}$ maps variables to rationals. We write $\Gamma(x)=q$ if the variable $x$ is bound to the value $q$ and $\Gamma(x)=\bot$ if unbound. The small-step semantics now depend on $\Gamma$. We write $\Gamma\vdash e \to e'$ if an expression $e$ steps (in one step) to $e'$ in environment $\Gamma$. 

We first define the rules that result in a constant:
\begin{mathpar}
\inferrule*[right=E-Var]{\Gamma(x)=q}{\Gamma\vdash \operatorname{var}(x) \to \operatorname{const}(q)}\\
\inferrule*[right=E-Add]{~}{\Gamma\vdash \operatorname{add}(\operatorname{const}(q_1),\operatorname{const}(q_2)) \to \operatorname{const}(q_1+q_2)}
\and
\inferrule*[right=E-Sub]{~}{\Gamma\vdash \operatorname{sub}(\operatorname{const}(q_1),\operatorname{const}(q_2)) \to \operatorname{const}(q_1-q_2)}\\
\inferrule*[right=E-Mul]{~}{\Gamma\vdash \operatorname{mul}(\operatorname{const}(q_1),\operatorname{const}(q_2)) \to \operatorname{const}(q_1\cdot q_2)}
\and
\inferrule*[right=E-Div]{q_2\neq 0}{\Gamma\vdash \operatorname{div}(\operatorname{const}(q_1),\operatorname{const}(q_2)) \to \operatorname{const}\!\left(\dfrac{q_1}{q_2}\right)}
\end{mathpar}


In addition we need the context rules. Evaluation order is left-to-right. 
\begin{mathpar}
\inferrule*[right=E-AddL]{\Gamma\vdash e_1 \to e_1'}{\Gamma\vdash \operatorname{add}(e_1,e_2) \to \operatorname{add}(e_1',e_2)}
\and
\inferrule*[right=E-AddR]{\Gamma\vdash e_2 \to e_2'}{\Gamma\vdash \operatorname{add}(\operatorname{const}(q),e_2) \to \operatorname{add}(\operatorname{const}(q),e_2')}\\

\inferrule*[right=E-SubL]{\Gamma\vdash e_1 \to e_1'}{\Gamma\vdash \operatorname{sub}(e_1,e_2) \to \operatorname{sub}(e_1',e_2)}
\and
\inferrule*[right=E-SubR]{ \Gamma\vdash e_2 \to e_2'}{\Gamma\vdash \operatorname{sub}(\operatorname{const}(q),e_2) \to \operatorname{sub}(\operatorname{const}(q),e_2')}\\

\inferrule*[right=E-MulL]{\Gamma\vdash e_1 \to e_1'}{\Gamma\vdash \operatorname{mul}(e_1,e_2) \to \operatorname{mul}(e_1',e_2)}
\and
\inferrule*[right=E-MulR]{ \Gamma\vdash e_2 \to e_2'}{\Gamma\vdash \operatorname{mul}(\operatorname{const}(q),e_2) \to \operatorname{mul}(\operatorname{const}(q),e_2')}\\

\inferrule*[right=E-DivL]{\Gamma\vdash e_1 \to e_1'}{\Gamma\vdash \operatorname{div}(e_1,e_2) \to \operatorname{div}(e_1',e_2)}
\and
\inferrule*[right=E-DivR]{\Gamma\vdash e_2 \to e_2'}{\Gamma\vdash \operatorname{div}(\operatorname{const}(q),e_2) \to \operatorname{div}(\operatorname{const}(q),e_2')}
\end{mathpar}



No rule applies for division by zero or an unbound variable, so those forms are stuck. This corresponds to the evaluation function that would return none. 

We write $\Gamma\vdash e \to^{\!*} e'$ for the reflexive-transitive closure of $\to$. To be more exact, $\Gamma \vdash e \to^{*}e'$ if either $e=e'$ (the reflexive case) or there exists an $e''$ such that $\Gamma \vdash e \to^{*} e''$ and $\Gamma\vdash e''\to e'$ (the transitive case).

\section{Helper lemmas}

\begin{lemma}[Concatenation, Lean: \texttt{Steps.append}] 
    \label{concatenation-lemma}
    If $\Gamma\vdash e\to^{\!*}e_1$ and $\Gamma\vdash e_1\to^{\!*}e_2$ then $\Gamma\vdash e\to^{\!*}e_2$. 
\end{lemma}

\begin{proof}
Let $h_1 : \Gamma\vdash e\to^{\!*}e_1$ and $h_2 : \Gamma\vdash e_1\to^{\!*}e_2$. We proceed by induction on $h_2$.

\textbf{Base case:} $h_2$ is the reflexive rule, so $e_1 = e_2$. Then $\Gamma\vdash e\to^{\!*}e_1 = e_2$ by $h_1$.

\textbf{Inductive case:} $h_2$ is of the form $\Gamma\vdash e_1\to^{\!*}e'$ followed by a single step $\Gamma\vdash e'\to e_2$ for some intermediate expression $e'$. 

By the induction hypothesis applied to $h_1$ and $\Gamma\vdash e_1\to^{\!*}e'$, we obtain $\Gamma\vdash e\to^{\!*}e'$. By definition of $\to^{*}$ we can combine $\Gamma\vdash e\to^{\!*}e'$ with the single step $\Gamma\vdash e'\to e_2$ to get $\Gamma\vdash e\to^{\!*}e_2$.
\end{proof}

\begin{lemma}[Context lifting, Lean: \texttt{Steps.liftCtx}] 
    \label{context-lifting-lemma}
    Let $C[\_]$ be a one-hole context formed by adding a fixed surrounding constructor (e.g. $C[t]=\operatorname{add}(t,e_2)$). Suppose every single step lifts through $C$: for all expressions $e_1, e_2$ whenever $\Gamma\vdash e_1\to e_2$ then $\Gamma\vdash C[e_1]\to C[e_2]$. Then if $\Gamma\vdash e\to^{\!*} e'$ we have $\Gamma\vdash C[e]\to^{\!*}C[e']$. 
\end{lemma}

\begin{proof}
Let $h : \Gamma\vdash e\to^{\!*}e'$ and assume the lifting property. We proceed by induction on $h$.

\textbf{Base case:} $h$ is the reflexive rule, so $e = e'$. Then $C[e] = C[e']$ and $\Gamma\vdash C[e]\to^{\!*}C[e']$ by reflexivity.

\textbf{Inductive case:} $h$ is of the form $\Gamma\vdash e\to^{\!*}e''$ followed by a single step $\Gamma\vdash e''\to e'$ for some intermediate expression $e''$.

By the induction hypothesis applied to $\Gamma\vdash e\to^{\!*}e''$, we obtain $\Gamma\vdash C[e]\to^{\!*}C[e'']$.

By the lifting property applied to the single step $\Gamma\vdash e''\to e'$, we obtain $\Gamma\vdash C[e'']\to C[e']$.

By definition of $\to^{*}$ we can combine $\Gamma\vdash C[e]\to^{\!*}C[e'']$ with $\Gamma\vdash C[e'']\to C[e']$ to get $\Gamma\vdash C[e]\to^{\!*}C[e']$.
\end{proof}


\begin{lemma}[Reduction lemmas, Lean: \texttt{Steps.reduceAdd}, \texttt{Steps.reduceSub}, \texttt{Steps.reduceMul}, \texttt{Steps.reduceDiv}] For each arithmetic operator we derive a multi-step reduction once its operands are reduced to constants:
\begin{align*}
&\text{Add: }\Gamma\vdash e_1\to^{\!*}\operatorname{const}(r_1),\ \Gamma\vdash e_2\to^{\!*}\operatorname{const}(r_2) \Rightarrow \Gamma\vdash \operatorname{add}(e_1,e_2)\to^{\!*}\operatorname{const}(r_1+r_2).\\
&\text{Sub: }\Gamma\vdash e_1\to^{\!*}\operatorname{const}(r_1),\ \Gamma\vdash e_2\to^{\!*}\operatorname{const}(r_2) \Rightarrow \Gamma\vdash \operatorname{sub}(e_1,e_2)\to^{\!*}\operatorname{const}(r_1-r_2).\\
&\text{Mul: }\Gamma\vdash e_1\to^{\!*}\operatorname{const}(r_1),\ \Gamma\vdash e_2\to^{\!*}\operatorname{const}(r_2)\Rightarrow \Gamma\vdash \operatorname{mul}(e_1,e_2)\to^{\!*}\operatorname{const}(r_1\cdot r_2).\\
&\text{Div: }r_2\neq 0,\ \Gamma\vdash e_1\to^{\!*}\operatorname{const}(r_1),\ \Gamma\vdash e_2\to^{\!*}\operatorname{const}(r_2) \Rightarrow \Gamma\vdash \operatorname{div}(e_1,e_2)\to^{\!*}\operatorname{const}(r_1/r_2).
\end{align*}
\end{lemma}

\begin{proof}
We proceed left-to-right by lifting each operand’s multi-step reduction through its evaluation context (Lemma \ref{context-lifting-lemma}) and then append the computation step using concatenation (Lemma \ref{concatenation-lemma}).
\begin{description}
\item[Add.] Let $s_1 : \Gamma\vdash e_1\to^{\!*}\operatorname{const}(r_1)$ and $s_2 : \Gamma\vdash e_2\to^{\!*}\operatorname{const}(r_2)$.

We apply the context lifting lemma (Lemma \ref{context-lifting-lemma}) to $s_1$ using $C_L[t]=\operatorname{add}(t,e_2)$ with lifting property E-AddL to obtain:
\[
L : \Gamma\vdash \operatorname{add}(e_1,e_2)\to^{\!*}\operatorname{add}(\operatorname{const}(r_1),e_2).
\]

We apply the context lifting again to $s_2$ using $C_R[t]=\operatorname{add}(\operatorname{const}(r_1),t)$ with lifting property E-AddR to obtain:
\[
R : \Gamma\vdash \operatorname{add}(\operatorname{const}(r_1),e_2)\to^{\!*}\operatorname{add}(\operatorname{const}(r_1),\operatorname{const}(r_2)).
\]

By rule E-Add, we have $\Gamma\vdash \operatorname{add}(\operatorname{const}(r_1),\operatorname{const}(r_2))\to \operatorname{const}(r_1+r_2)$. We extend $R$ with this step by the definition of $\to^{*}$ to get:
\[
R' : \Gamma\vdash \operatorname{add}(\operatorname{const}(r_1),e_2)\to^{\!*}\operatorname{const}(r_1+r_2).
\]

Applying concatenation lemma (Lemma \ref{concatenation-lemma}) to $L$ and $R'$ yields $\Gamma\vdash \operatorname{add}(e_1,e_2)\to^{\!*}\operatorname{const}(r_1+r_2)$.

\item[Sub/Mul.] Identical reasoning to addition using E-SubL/E-SubR/E-Sub and E-MulL/E-MulR/E-Mul respectively.

\item[Div.] Identical reasoning to addition using E-DivL/E-DivR, with the premise $r_2\neq 0$ required to apply the E-Div computation rule in the final step.
\end{description}
\end{proof}


\newpage
\section{Proofs}

For the upcoming proofs, recall the definitions of the evaluation function from \texttt{math-expr.lean}:
\begin{align*}
&\operatorname{eval}(\operatorname{const}(q),\Gamma)=\text{some }q,\\
&\operatorname{eval}(\operatorname{var}(x),\Gamma)=\Gamma[x]?,\\
&\operatorname{eval}(\operatorname{add}(a,b),\Gamma)=\text{do }(\leftarrow \operatorname{eval}(a,\Gamma)) + (\leftarrow \operatorname{eval}(b,\Gamma)),\\
&\operatorname{eval}(\operatorname{sub}(a,b),\Gamma)=\text{do }(\leftarrow \operatorname{eval}(a,\Gamma)) - (\leftarrow \operatorname{eval}(b,\Gamma)),\\
&\operatorname{eval}(\operatorname{mul}(a,b),\Gamma)=\text{do }(\leftarrow \operatorname{eval}(a,\Gamma)) * (\leftarrow \operatorname{eval}(b,\Gamma)),\\
&\operatorname{eval}(\operatorname{div}(a,b),\Gamma)=\text{do }r_b\leftarrow \operatorname{eval}(b,\Gamma);\ \textbf{if }r_b=0\textbf{ then }\text{none}\textbf{ else }r_a\leftarrow \operatorname{eval}(a,\Gamma);\ \text{some }(r_a/r_b).
\end{align*}

\subsection*{Equivalence}

We say an expression $e$ is \emph{stuck} if is not of the form $\operatorname{const}(q)$ for some $q\in\mathbb{Q}$ and it cannot take a step: there is no $e'$ with $\Gamma\vdash e\to e'$. This includes both terminal stuck forms and compound expressions containing them (e.g., $\operatorname{add}(\operatorname{var}(x),e_2)$ where $\Gamma(x)=\bot$).

To show that the evaluation function and the small step semantics agree, we have to show that
\[ \operatorname{eval}(e, \Gamma) = \operatorname{some} q \iff \Gamma\vdash e \to^{\!*} \operatorname{const}(q)
\]
and 
\[ \operatorname{eval}(e, \Gamma) = \text{none} \iff \Gamma\vdash e \to^{\!*} n \text{ stuck as above}.\]

We next prove that single steps preserve the result of \texttt{eval}, and lift this to multi-step preservation. From multi-step preservation we can then conclude soundness (small-step reducing to a constant implies that the evaluation function evaluates to the same value). Then we establish completeness, showing any successful evaluation yields a multi-step reduction to the corresponding constant. Finally, we analyze the failure cases to complete the second equivalence, proving that evaluation yields none if and only if the small-step semantics reduces the expression to a stuck term.


\subsection*{Evaluation preservation}
\begin{lemma}[Single-step preserves evaluation, Lean: \texttt{step\_preserves\_eval}]
    \label{single-step-preservation-lemma}
If $\Gamma\vdash e\to e'$ then $\operatorname{eval}(e,\Gamma)=\operatorname{eval}(e',\Gamma)$.
\end{lemma}
\begin{proof} We do induction on $\Gamma\vdash e\to e'$. Cases:


\begin{description}
\item[E-Var.] By definition of the rule we have $\Gamma(x)=q$. Then
\begin{align}
\operatorname{eval}(\operatorname{var}(x),\Gamma)&=\Gamma[x]? \tag{definition of eval}\nonumber\\
&=\text{some }q \tag{since $\Gamma(x)=q$}\nonumber\\
&=\operatorname{eval}(\operatorname{const}(q),\Gamma). \tag{definition of eval}\nonumber
\end{align}


\item [E-AddL.] By definition of the rule we have $\Gamma\vdash e_1\to e_1'$. Applying the induction hypothesis, this gives $\operatorname{eval}(e_1,\Gamma)=\operatorname{eval}(e_1',\Gamma)$. Then
\begin{align}
\operatorname{eval}(\operatorname{add}(e_1,e_2),\Gamma)
&=\text{do }(\leftarrow \operatorname{eval}(e_1,\Gamma)) + (\leftarrow \operatorname{eval}(e_2,\Gamma)) \tag{definition of eval}\nonumber\\
&=\text{do }(\leftarrow \operatorname{eval}(e_1',\Gamma)) + (\leftarrow \operatorname{eval}(e_2,\Gamma)) \tag{$\operatorname{eval}(e_1,\Gamma)=\operatorname{eval}(e_1',\Gamma)$}\nonumber\\
&=\operatorname{eval}(\operatorname{add}(e_1',e_2),\Gamma).\tag{definition of eval}\nonumber
\end{align}


\item [E-AddR.] By definition of the rule we have that the first operand is of the form $\operatorname{const}(q)$ for some $q$ and that $\Gamma\vdash e_2\to e_2'$. Applying the induction hypothesis, this gives $\operatorname{eval}(e_2,\Gamma)=\operatorname{eval}(e_2',\Gamma)$. Then
\begin{align}
\operatorname{eval}(\operatorname{add}(\operatorname{const}(q),e_2),\Gamma)
&=\text{do }(\leftarrow \operatorname{eval}(\operatorname{const}(q),\Gamma)) + (\leftarrow \operatorname{eval}(e_2,\Gamma)) \tag{definition of eval}\nonumber\\
&=\text{do }(\leftarrow \operatorname{eval}(\operatorname{const}(q),\Gamma)) + (\leftarrow \operatorname{eval}(e_2',\Gamma))\tag{$\operatorname{eval}(e_2,\Gamma)=\operatorname{eval}(e_2',\Gamma)$}\nonumber\\
&=\operatorname{eval}(\operatorname{add}(\operatorname{const}(q),e_2'),\Gamma). \tag{definition of eval}\nonumber
\end{align}


\item [E-Add.] By definition of the rule we have that both operands are of the form $\operatorname{const}(q)$ for some $q$. Then
\begin{align*}
\operatorname{eval}(\operatorname{add}(\operatorname{const}(q_1),\operatorname{const}(q_2)),\Gamma)
&=\text{do }(\leftarrow \text{some }q_1) + (\leftarrow \text{some }q_2) \tag{definition of eval}\nonumber\\
&=\text{some }(q_1+q_2) \tag{evaluate}\nonumber\\
&=\operatorname{eval}(\operatorname{const}(q_1+q_2),\Gamma). \tag{definition of eval}\nonumber
\end{align*}


\item [E-SubL/E-SubR/E-Sub.] Identical equational reasoning with $-$ in place of $+$.


\item [E-MulL/E-MulR/E-Mul.] Identical equational reasoning with $*$ in place of $+$.


\item [E-DivL.] By definition of the rule we have $\Gamma\vdash e_1\to e_1'$ and applying the induction hypothesis gives $\operatorname{eval}(e_1,\Gamma)=\operatorname{eval}(e_1',\Gamma)$. Then
\begin{align*}
\operatorname{eval}(\operatorname{div}(e_1,e_2),\Gamma)
&=\text{do }r_b\leftarrow \operatorname{eval}(e_2,\Gamma);\ \textbf{if }r_b=0\textbf{ then }\text{none}\tag{definition of eval}\nonumber\\
&\qquad\textbf{ else }r_a\leftarrow \operatorname{eval}(e_1,\Gamma);\ \text{some }(r_a/r_b)\\
&=\text{do }r_b\leftarrow \operatorname{eval}(e_2,\Gamma);\ \textbf{if }r_b=0\textbf{ then }\text{none}\\
&\qquad\textbf{ else }r_a\leftarrow \operatorname{eval}(e_1',\Gamma);\ \text{some }(r_a/r_b) \tag{$\operatorname{eval}(e_1,\Gamma)=\operatorname{eval}(e_1',\Gamma)$}\nonumber\\
&=\operatorname{eval}(\operatorname{div}(e_1',e_2),\Gamma). \tag{definition of eval}\nonumber
\end{align*}


\item [E-DivR.] By definition of the rule we have that the first operand is of the form $\operatorname{const}(q)$ for some $q$ and that $\Gamma\vdash e_2\to e_2'$. Applying the induction hypothesis gives $\operatorname{eval}(e_2,\Gamma)=\operatorname{eval}(e_2',\Gamma)$. Then
\begin{align*}
\operatorname{eval}(\operatorname{div}(\operatorname{const}(q),e_2),\Gamma)
&=\text{do }r_b\leftarrow \operatorname{eval}(e_2,\Gamma);\ \textbf{if }r_b=0\textbf{ then }\text{none}\\
&\qquad\textbf{ else }r_a\leftarrow \operatorname{eval}(\operatorname{const}(q),\Gamma);\ \text{some }(r_a/r_b)\tag{definition of eval}\nonumber\\
&=\text{do }r_b\leftarrow \operatorname{eval}(e_2',\Gamma);\ \textbf{if }r_b=0\textbf{ then }\text{none}\\
&\qquad\textbf{ else }r_a\leftarrow \operatorname{eval}(\operatorname{const}(q), \Gamma);\ \text{some }(r_a/r_b)\tag{$\operatorname{eval}(e_2,\Gamma)=\operatorname{eval}(e_2',\Gamma)$}\nonumber\\
&=\operatorname{eval}(\operatorname{div}(\operatorname{const}(q),e_2'),\Gamma).\tag{definition of eval}\nonumber
\end{align*}

\item [E-Div.] By definition of the rule we have that both operands are of the form $\operatorname{const}(q)$ for some $q$ and that the second constant is nonzero. Then
\begin{align*}
\operatorname{eval}(\operatorname{div}(\operatorname{const}(q_1),\operatorname{const}(q_2)),\Gamma)
&=\text{do }r_b\leftarrow \operatorname{eval}(\operatorname{const}(q_2), \Gamma); \textbf{if }r_b=0\textbf{ then }\text{none } \\ 
& \qquad\textbf{ else } r_a\leftarrow \text{some }q_1;\ \text{some }(r_a/r_b)\tag{definition of eval}\nonumber\\
&=\text{some }(q_1/q_2) \tag{evaluate with $q_2\neq 0$}\nonumber\\
&=\operatorname{eval}(\operatorname{const}(q_1/q_2),\Gamma).\tag{definition of eval}\nonumber
\end{align*}
\end{description}

Thus, case analysis over all possible derivation rules shows in every instance that \(\operatorname{eval}(e,\Gamma)=\operatorname{eval}(e',\Gamma)\). Hence single-step reduction preserves evaluation, completing the proof.
\end{proof}

\begin{lemma}[Multi-step preserves evaluation (Lean: \texttt{steps\_preserve\_eval})]
If $\Gamma\vdash e\to^{\!*} e'$ then $\operatorname{eval}(e,\Gamma)=\operatorname{eval}(e',\Gamma)$.
\end{lemma}

\begin{proof}
We proceed by induction on the derivation of $\Gamma\vdash e\to^{\!*} e'$.

\textbf{Base case (reflexive):} The derivation is the reflexive rule, so $e = e'$. Then $\operatorname{eval}(e,\Gamma)=\operatorname{eval}(e',\Gamma)$ holds trivially.

\textbf{Inductive case (transitive):} By definition of $\to^{*}$, the derivation has the form $\Gamma\vdash e\to^{\!*}e''$ followed by a single step $\Gamma\vdash e''\to e'$ for some intermediate expression $e''$. 

By the induction hypothesis applied to $\Gamma\vdash e\to^{\!*}e''$, we obtain $\operatorname{eval}(e,\Gamma)=\operatorname{eval}(e'',\Gamma)$.

By the single-step preservation lemma (Lemma \ref{single-step-preservation-lemma}) applied to $\Gamma\vdash e''\to e'$, we obtain $\operatorname{eval}(e'',\Gamma)=\operatorname{eval}(e',\Gamma)$.

By transitivity of equality, combining these two results yields $\operatorname{eval}(e,\Gamma)=\operatorname{eval}(e',\Gamma)$.

Thus multi-step reduction preserves evaluation.
\end{proof}



\subsection*{Soundness}
\begin{theorem}[Soundness, Lean: \texttt{small\_step\_soundness}]
If $\Gamma\vdash e \to^{\!*} \operatorname{const}(q)$ then $\operatorname{eval}(e, \Gamma) = \operatorname{some} q$.
\end{theorem}
\begin{proof}
By multi-step preservation and the first clause of the definition of \texttt{eval},
\begin{align*}
\operatorname{eval}(e,\Gamma)&=\operatorname{eval}(\operatorname{const}(q),\Gamma)\\
&=\text{some }q
\end{align*}
\end{proof}

%\newpage
\subsection*{Completeness}



\begin{theorem}[Completeness (Lean: \texttt{eval\_some\_implies\_steps\_to\_const})]
    \label{completeness-theorem}
If $\operatorname{eval}(e, \Gamma) = \operatorname{some}q$ then $\Gamma\vdash e \to^{\!*} \operatorname{const}(q)$.
\end{theorem}
\begin{proof}
By functional induction on the evaluation function:
\begin{description}
\item[Const.] We have $e = \operatorname{const}(r)$ for some $r\in \mathbb{Q}$. Then, $\operatorname{eval}(\operatorname{const}(r),\Gamma)=\text{some }q$. By definition of eval, we have $r = q$. By taking zero steps we have $\Gamma \vdash \operatorname{const}(r) \to^{*} \operatorname{const}(q)$.

\item[Var.] We have $e = \operatorname{var}(x)$ for some string $x$. Then, $\operatorname{eval}(\operatorname{var}(x),\Gamma)=\text{some }q$. By definition of eval we have $\Gamma[x]?=\text{some }q$ and thus $\Gamma(x) = q$. By applying E-Var once we get $\Gamma\vdash \operatorname{var}(x)\to^{*}\text{const }q$.

\item[Add.] We have $e = \operatorname{add}(e_1,e_2)$.

Our assumption
\begin{align*}
\operatorname{eval}(\operatorname{add}(e_1,e_2),\Gamma)=\text{some }q
\end{align*}
unfolds to
\begin{align*}
\text{do }(\leftarrow \operatorname{eval}(e_1,\Gamma)) + (\leftarrow \operatorname{eval}(e_2,\Gamma))&=\text{some }q,
\end{align*}
which is possible iff there exist $r_1,r_2\in \mathbb{Q}$ with
\begin{align*}
\operatorname{eval}(e_1,\Gamma)=\text{some }r_1,\quad \operatorname{eval}(e_2,\Gamma)=\text{some }r_2,\quad q=r_1+r_2.
\end{align*}
We apply the induction hypothesis to both eval-expressions to obtain $\Gamma\vdash e_1\to^{\!*}\operatorname{const}(r_1)$ and $\Gamma\vdash e_2\to^{\!*}\operatorname{const}(r_2)$. We apply the reduction lemma to obtain
$\Gamma\vdash \operatorname{add}(e_1,e_2)\to^{\!*}\operatorname{const}(r_1+r_2)$.
\item[Sub/Mul.] Identical unpacking of the monadic definition with $-$ or $*$, yielding $q=r_1-r_2$ or $q=r_1*r_2$ by applying the respective reduction lemma.

\item[Div.] We have $e = \operatorname{div}(e_1,e_2)$. 

Our assumption
\begin{align*}
\operatorname{eval}(\operatorname{div}(e_1,e_2),\Gamma)=\text{some }q
\end{align*}
unfolds to
\begin{align*}
\text{do }r_2\leftarrow \operatorname{eval}(e_2,\Gamma);\ \textbf{if }r_2=0\textbf{ then }\text{none}\textbf{ else }r_1\leftarrow \operatorname{eval}(e_1,\Gamma);\ \text{some }(r_1/r_2)=\text{some }q,
\end{align*}
which is possible iff there exist $r_1,r_2\in \mathbb{Q}$ with
\begin{align*}
\operatorname{eval}(e_2,\Gamma)=\text{some }r_2,\ r_2\neq 0,\ \operatorname{eval}(e_1,\Gamma)=\text{some }r_1,\ q=r_1/r_2.
\end{align*}
We apply our induction hypothesis to both evaluations to obtain $\Gamma\vdash e_1\to^{\!*}\operatorname{const}(r_1)$ and $\Gamma\vdash e_2\to^{\!*}\operatorname{const}(r_2)$, then the division reduction lemma (using $r_2\neq 0$) yields $\Gamma\vdash\operatorname{div}(e_1, e_2)\to^{*}\operatorname{const}(r_1/r_2)$.
\end{description}
\end{proof}

% \newpage
\subsection*{Failure and stuckness}

\begin{theorem}
$\operatorname{eval}(e,\Gamma)=\text{none}$ iff $\Gamma\vdash e \to^{\!*} n$ where $n$ is stuck.
\end{theorem}

\begin{proof}
We prove both directions.

\paragraph{($\Rightarrow$) If $\operatorname{eval}(e,\Gamma)=\text{none}$ then $\Gamma\vdash e \to^{\!*} n$ where $n$ is stuck.}

By functional induction on the evaluation function.

\begin{description}
\item[Const.] We have $e=\operatorname{const}(q)$. Then $\operatorname{eval}(\operatorname{const}(q),\Gamma)=\text{some }q\neq\text{none}$, contradiction.

\item[Var.] We have $e=\operatorname{var}(x)$. Then $\operatorname{eval}(\operatorname{var}(x),\Gamma)=\Gamma[x]?=\text{none}$ means $\Gamma(x)=\bot$. We have $\Gamma\vdash e\to^{\!*}e$ by zero steps (reflexivity). The expression $\operatorname{var}(x)$ with $\Gamma(x)=\bot$ is stuck: no rule applies since E-Var requires $\Gamma(x)=q$ for some $q$.

\item[Add.] We have $e=\operatorname{add}(e_1,e_2)$. The monadic definition gives
\[
\operatorname{eval}(\operatorname{add}(e_1,e_2),\Gamma)=\text{do }(\leftarrow \operatorname{eval}(e_1,\Gamma)) + (\leftarrow \operatorname{eval}(e_2,\Gamma))=\text{none}.
\]
This is possible iff $\operatorname{eval}(e_1,\Gamma)=\text{none}$ or $\operatorname{eval}(e_2,\Gamma)=\text{none}$ (or both).

\emph{Subcase} $\operatorname{eval}(e_1,\Gamma)=\text{none}$. By induction hypothesis, $\Gamma\vdash e_1\to^{\!*}n_1$ where $n_1$ is stuck. By context lifting (using E-AddL repeatedly), $\Gamma\vdash \operatorname{add}(e_1,e_2)\to^{\!*}\operatorname{add}(n_1,e_2)$. Since $n_1$ is stuck and thus cannot be of the form $\operatorname{const}(q)$ for any $q\in\mathbb{Q}$ both E-AddR and E-Add cannot apply. Since $n_1$ is stuck and thus $\not\exists e'. \Gamma\vdash n_1\to e'$, E-AddL also cannot apply. 

\emph{Subcase} $\operatorname{eval}(e_1,\Gamma)=\text{some }r_1$ and $\operatorname{eval}(e_2,\Gamma)=\text{none}$. By completeness (Theorem \ref{completeness-theorem}), $\Gamma\vdash e_1\to^{\!*}\operatorname{const}(r_1)$. By induction hypothesis on $e_2$, $\Gamma\vdash e_2\to^{\!*}n_2$ where $n_2$ is stuck. Context lifting gives $\Gamma\vdash \operatorname{add}(e_1,e_2)\to^{\!*}\operatorname{add}(\operatorname{const}(r_1),e_2)\to^{\!*}\operatorname{add}(\operatorname{const}(r_1),n_2)$. This is stuck: E-Add requires both arguments to be constants, but $n_2$ is not a constant; E-AddR requires $n_2$ to step, but $n_2$ is stuck.

\item[Sub/Mul.] We have $e=\operatorname{sub}(e_1,e_2)$, $e=\operatorname{mul}(e_1,e_2)$. Identical reasoning to addition using E-SubL/E-SubR/E-Sub and E-MulL/E-MulR/E-Mul respectively.

\item[Div.] We have $e=\operatorname{div}(e_1,e_2)$. The definition is
\[
\operatorname{eval}(\operatorname{div}(e_1,e_2),\Gamma)=\text{do }r_2\leftarrow \operatorname{eval}(e_2,\Gamma);\ \textbf{if }r_2=0\textbf{ then }\text{none}\textbf{ else }r_1\leftarrow \operatorname{eval}(e_1,\Gamma);\ \text{some }(r_1/r_2).
\]
This equals none iff:
\begin{itemize}
\item $\operatorname{eval}(e_2,\Gamma)=\text{none}$, or
\item $\operatorname{eval}(e_2,\Gamma)=\text{some }r_2$ with $r_2=0$ and any outcome for $e_1$, or
\item $\operatorname{eval}(e_2,\Gamma)=\text{some }r_2$ with $r_2\neq 0$ and $\operatorname{eval}(e_1,\Gamma)=\text{none}$.
\end{itemize}

\emph{Subcase} $\operatorname{eval}(e_2,\Gamma)=\text{none}$. By induction hypothesis, $\Gamma\vdash e_2\to^{\!*}n_2$ stuck. Context lifting gives $\Gamma\vdash \operatorname{div}(e_1,e_2)\to^{\!*}\operatorname{div}(e_1,n_2)$, which is stuck by similar reasoning to addition.

\emph{Subcase} $\operatorname{eval}(e_2,\Gamma)=\text{some }0$ and $\operatorname{eval}(e_1,\Gamma)=\text{some }r_1$. By completeness, $\Gamma\vdash e_1\to^{\!*}\operatorname{const}(r_1)$ and $\Gamma\vdash e_2\to^{\!*}\operatorname{const}(0)$. Context lifting yields $\Gamma\vdash \operatorname{div}(e_1,e_2)\to^{\!*}\operatorname{div}(\operatorname{const}(r_1),\operatorname{const}(0))$. This is stuck: E-Div requires $q_2\neq 0$, which fails here.

\emph{Subcase} $\operatorname{eval}(e_2,\Gamma)=\text{some }0$ and $\operatorname{eval}(e_1,\Gamma)=\text{none}$. By induction hypothesis, $\Gamma\vdash e_1\to^{\!*}n_1$ stuck. By completeness on $e_2$, $\Gamma\vdash e_2\to^{\!*}\operatorname{const}(0)$. We reach $\Gamma\vdash \operatorname{div}(e_1,e_2)\to^{\!*}\operatorname{div}(n_1,\operatorname{const}(0))$, which is stuck.

\emph{Subcase} $\operatorname{eval}(e_2,\Gamma)=\text{some }r_2$ with $r_2\neq 0$ and $\operatorname{eval}(e_1,\Gamma)=\text{none}$. By completeness on $e_2$ and induction hypothesis on $e_1$, we reach $\Gamma\vdash \operatorname{div}(e_1,e_2)\to^{\!*}\operatorname{div}(n_1,\operatorname{const}(r_2))$, which is stuck.
\end{description}

\paragraph{($\Leftarrow$) If $\Gamma\vdash e \to^{\!*} n$ where $n$ is stuck, then $\operatorname{eval}(e,\Gamma)=\text{none}$.}

By multi-step preservation, $\operatorname{eval}(e,\Gamma)=\operatorname{eval}(n,\Gamma)$. We show $\operatorname{eval}(n,\Gamma)=\text{none}$ by functional induction on the evaluation function.

\begin{description}
    \item[Const.] We have $n=\operatorname{const}(q)$. By definition they are not stuck.
    \item[Var.] We have $n=\operatorname{var}(x)$ with $\Gamma(x)=\bot$. Then $\operatorname{eval}(\operatorname{var}(x),\Gamma)=\Gamma[x]?=\text{none}$ by definition.
    \item[Add. ] We have $n=\operatorname{add}(n_1,n_2)$ stuck. Since $n$ is stuck, E-AddL cannot apply, so $n_1$ must be stuck. By induction hypothesis, $\operatorname{eval}(n_1,\Gamma)=\text{none}$. Then
    \[
    \operatorname{eval}(\operatorname{add}(n_1,n_2),\Gamma)=\text{do }(\leftarrow \operatorname{eval}(n_1,\Gamma)) + (\leftarrow \operatorname{eval}(n_2,\Gamma))=\text{do }(\leftarrow \text{none}) + \cdots=\text{none}.
    \]
    \item[Sub. / Mul. ] We have either $n=\operatorname{sub}(n_1,n_2)$ or $n=\operatorname{mul}(n_1,n_2)$ stuck. Identical reasoning: the left operand must be stuck, so evaluation returns none.    

    \item[Div. ] We have $n=\operatorname{div}(n_1,n_2)$ stuck. Since evaluation is left-to-right, for $n$ to be stuck we need E-DivL, E-DivR, and E-Div all to fail.
    
    E-DivL fails means $n_1$ cannot step, so either $n_1$ is stuck (and not a constant) or $n_1=\operatorname{const}(r_1)$ for some $r_1$.
    
    \emph{Case 1:} $n_1$ is stuck (not a constant). Then by induction hypothesis $\operatorname{eval}(n_1,\Gamma)=\text{none}$, giving
    \[
    \operatorname{eval}(\operatorname{div}(n_1,n_2),\Gamma)=\text{do }r_2\leftarrow \operatorname{eval}(n_2,\Gamma);\ \textbf{if }r_2=0\textbf{ then }\text{none}\textbf{ else }r_1\leftarrow \text{none};\cdots=\text{none}.
    \]
    
    \emph{Case 2:} $n_1=\operatorname{const}(r_1)$. Now E-DivR and E-Div must also fail. E-DivR fails means $n_2$ cannot step, so either $n_2$ is stuck or $n_2=\operatorname{const}(r_2)$.
    \begin{itemize}
    \item If $n_2$ is stuck (not a constant), then by induction hypothesis $\operatorname{eval}(n_2,\Gamma)=\text{none}$, giving
    \[
    \operatorname{eval}(\operatorname{div}(\operatorname{const}(r_1),n_2),\Gamma)=\text{do }r_2\leftarrow \text{none};\cdots=\text{none}.
    \]
    \item If $n_2=\operatorname{const}(r_2)$, then for E-Div to fail we need $r_2=0$. Then
    \[
    \operatorname{eval}(\operatorname{div}(\operatorname{const}(r_1),\operatorname{const}(0)),\Gamma)=\text{do }r_2\leftarrow \text{some }0;\ \textbf{if }r_2=0\textbf{ then }\text{none}\cdots=\text{none}.
    \]
    \end{itemize}
\end{description}

Thus every stuck expression evaluates to none.
\end{proof}



\end{document}